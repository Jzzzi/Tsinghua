\documentclass[UTF8]{ctexart}
\usepackage{amsmath}
\usepackage{amssymb}
\usepackage{cases}
\usepackage{cite}
\usepackage{graphicx}
\usepackage[margin=1in]{geometry}
\geometry{a4paper}
\usepackage{lmodern}%处理字体问题

\title{连续介质力学作业七}
\author{刘锦坤\\2022013352}
\date{\today}

\begin{document}

\maketitle

\section{第一题}

\section{第二题}
抱歉,我坚持认为存在关系
\begin{equation*}
    \frac{d \epsilon_{ij}}{dt}=e_{ij}  
\end{equation*}
我可以证明如下,
\begin{align*}
    e_{ij}&=(\frac{\partial \epsilon_{ij}}{\partial t})_{\xi^\alpha,t}\\
    &=(\frac{\partial \epsilon_{ij}}{\partial t})_{x^\alpha,t}
    +(\frac{\partial \epsilon_{ij}}{\partial x^i})_{x^\alpha,t}(\frac{\partial x^i}{\partial t})_{\xi^\alpha,t}
\end{align*}
其中括号后的下标表示求偏导时哪些是构成函数关系的变量,这完全就是数学上的复合映射的求导法则。且注意到
$(\frac{\partial x^i}{\partial t})_{\xi^\alpha,t}$就是$v_i$,因此上式完全就是物质导数的定义。

因此我认为
\begin{equation*}
    \frac{d \epsilon_{ij}}{dt}=e_{ij}  
\end{equation*}
一定成立。

\section{第三题}

先证明平均方向应力公式,即
\begin{equation*}
    \lim_{a\to 0}\frac{1}{4\pi a^2}\int_{S_a} p_{nn} d\sigma
    =\lim_{a\to 0}\frac{1}{4\pi a^2}\int_{S_a} \vec n \cdot \tilde P \cdot \vec n d\sigma
    =\frac{1}{3}p^{i\cdot}_{\cdot i}
    \notag
\end{equation*}
在考虑的空间点为原点,引入一个直角坐标系$oxyz$,且
应力张量的三个主方向为$\vec e_x,\vec e_y,\vec e_z$,
即在考虑的空间点处有
\begin{equation*}
    \tilde P=p_x \vec e_x \vec e_x + p_y \vec e_y \vec e_y + p_z \vec e_z \vec e_z
\end{equation*}
引入球坐标$\theta,\phi$进行积分计算,
有$\vec n = sin\theta cos\phi \vec e_x+ sin\theta sin\phi \vec e_y +cos\theta \vec e_z$,
带入得
\begin{align*}
    \lim_{a\to 0}\frac{1}{4\pi a^2}\int_{S_a} \vec n \cdot \tilde P \cdot \vec n d\sigma
    &=\int_{S_a} \lim_{a\to 0}\frac{\vec n \cdot \tilde P \cdot \vec n}{4\pi a^2} d\sigma\\
    &=\frac{1}{4\pi } \iint_{S_a} p_x sin^3\theta cos^2\phi + p_y sin^3\theta sin^2\phi + p_z sin\theta cos^2\theta d\theta d\phi\\
    &=\frac{1}{4\pi } \int_{0}^{2\pi} d\phi \int_{0}^{\pi} p_x sin^3\theta cos^2\phi + p_y sin^3\theta sin^2\phi + p_z sin\theta cos^2\theta d\theta \\
    &=\frac{p_x+p_y+p_z}{3}\\
    &=\frac{1}{3}p^{i\cdot}_{\cdot i}
\end{align*}
然后带入
\begin{equation*}
    p^{ij}=-pg^{ij}+\tau ^{ij}
    =-pg^{ij}+\lambda e^{\alpha\cdot}_{\cdot \alpha} g^{ij} +2 \mu e^{ij}
\end{equation*}
并且利用
\begin{equation*}
    tr(\tilde g)=3,\quad tr(\tilde e)=\nabla \cdot \vec v
\end{equation*}
即得
\begin{equation*}
    \lim_{a\to 0}\frac{1}{4\pi a^2}\int_{S_a} p_{nn} d\sigma
    =\frac{1}{3}p^{i\cdot}_{\cdot i}=-p+(\lambda + \frac{2\mu}{3})\nabla \cdot \vec v
\end{equation*}

\section{第四题}

接下来都在Euler观点下考虑。以下标表示各组分,组分$\alpha$的密度场记为$\rho_\alpha$,
速度场记为$\vec v_\alpha$,应力张量设为$\tilde P_\alpha$,受的质量力为$\vec F_\alpha$。
同时引入一个新的参量$\vec f_{\alpha\beta}$,
表示单位体积内$\alpha$组分对$\beta$组分的作用力。
那么取空间中固定的体积$V$,对于组分$\alpha$有
\begin{equation}
    \int_V \frac{\partial (\rho_\alpha\vec v_\alpha)}{\partial t} dV
    =\int_V \sum_{\beta\neq\alpha} \vec f_{\beta\alpha} dV
    +\int_V \rho_\alpha \vec F_\alpha dV
    +\int_{\partial V} \vec n \cdot \sum_{\beta}\tilde P_\beta  d\sigma
    +\int_{\partial V} (\vec n \cdot \vec v_\alpha) \rho_\alpha \vec v_\alpha d\sigma
    \tag{1}
\end{equation}
\noindent 由于不考虑化学反应,所以可以带入质量守恒方程
\begin{equation}
    \frac{\partial \rho_\alpha}{\partial t}+\nabla\cdot(\rho_\alpha\vec v_\alpha)=0
\end{equation}
\noindent 并且注意到$V$的任意性即得
\begin{equation}
    \rho_\alpha \frac{d \vec v_\alpha}{d t}
    -\sum_{\beta\neq\alpha} \vec f_{\beta\alpha}
    -\rho_\alpha \vec F_\alpha
    -\nabla\cdot\sum_\beta \tilde P_\beta
    =0
\end{equation}
\noindent 这即为组分$\alpha$的动量守恒方程。

\end{document}