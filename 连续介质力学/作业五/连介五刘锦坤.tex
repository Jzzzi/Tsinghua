\documentclass[UTF8]{ctexart}
\usepackage{amsmath}
\usepackage{cases}
\usepackage{cite}
\usepackage{graphicx}
\usepackage[margin=1in]{geometry}
\usepackage{lmodern}
\geometry{a4paper}
\usepackage{fancyhdr}
\pagestyle{fancy}
\fancyhf{}

\title{连续介质力学作业五}
\author{刘锦坤\\2022013352}
\date{\today}
\pagenumbering{arabic}

\begin{document}

\maketitle

\section{第一题}

为了便于表述,在本题之后的推导中,所有不加额外说明的求偏导默认对于Euler坐标系中的坐标进行,
对于Language坐标系中的坐标求偏导时,会进行额外说明,
例如$(\frac{\partial A}{\partial t})_{\xi^\alpha}$
表示物理量$A$在Language坐标系中对时间$t$求偏导。

对于速度和加速度,我们已经知道有
$$
\vec v = v^i\vec{e_i}
$$
\noindent 于是根据加速度的定义有
\begin{align*}
\vec a &= \frac{d\vec v}{dt} = (\frac{\partial \vec v}{\partial t})_{\xi^\alpha}
=(\frac {\partial (v^i \vec e_i)}{\partial t})_{\xi^\alpha}\\
&=(\frac{\partial v^i}{\partial t})_{\xi^\alpha} \vec e_i+
v^i (\frac{\partial \vec e_i}{\partial t})_{\xi^\alpha}\\
&=\frac {dv^i}{dt} \vec e_i
+v^i(
    \frac{\partial \vec e_i}{\partial t}
    +\frac{\partial \vec e_i}{\partial x^j} (\frac{\partial x^j}{\partial t})_{\xi^\alpha}
    )
\end{align*}
\noindent 由于Euler坐标系是静止的,故$\frac{\partial \vec e_i}{\partial t}=0$,
且注意到
$$
\frac{\partial \vec e_i}{\partial x_j}=\Gamma_{ij}^k \vec e_k,
(\frac{\partial x^j}{\partial t})_{\xi^\alpha}=v^j
$$
\noindent 于是有
$$
\vec a = \frac {dv^i}{dt} \vec e_i + v^i v^j \Gamma_{ij}^k \vec e_k
$$
\noindent 故得到
\begin{equation}
a^k=\frac {dv^k}{dt} + v^i v^j \Gamma_{ij}^k
\tag{1}
\end{equation}

对于应变张量和应变率张量,根据应变率张量的定义即有
\begin{equation}
e^{ij}=(\frac{\partial \epsilon^{ij}}{\partial t})_{\xi^\alpha}=\frac{d\epsilon^{ij}}{dt}
\tag{2}
\end{equation}

\section{第二题}

接下来的讨论都在Euler观点下进行,记A,B,C三种物质的密度场为
$\rho_A,\rho_B,\rho_C$,速度场记为$\vec v_A,\vec v_B,\vec v_C$,
单位时间单位体积内A反应的质量记为$X$,则B反应的质量则可记为$2X$,
生成的$C$的质量则可记为$X$,则有
\noindent 对于Euler坐标下的固定体积$V$,有
$$
\frac{d}{dt}\int_V (\rho_A+\rho_B+\rho_C) dV
+\int_{\partial V} (\rho_A \vec v_A+\rho_B \vec v_B+\rho_C \vec v_C) \cdot d\vec\sigma=0
$$
\noindent 应用Gauss积分公式,并且考虑到体积$V$的任意性即得
\begin{equation}
\frac {\partial (\rho_A+\rho_B+\rho_C)}{\partial t}
+\nabla \cdot (\rho_A \vec v_A+\rho_B \vec v_B+\rho_C \vec v_C)
=0
\tag{1}
\end{equation}
\noindent 那么对于物质A可以列出
$$
\frac{d}{dt} \int_V \rho_A dV
+\int_{\partial V} \rho_A \vec v_A \cdot d\vec\sigma
+\int_V X dV
=0
$$
\noindent 应用Gauss公式,并且考虑到体积V的任意性即得
$$
\frac{\partial \rho_A}{\partial t} + \nabla \cdot (\rho_A \vec v_A) = -X
$$
\noindent 同理对B可得
$$
\frac{\partial \rho_B}{\partial t} + \nabla \cdot (\rho_B \vec v_B) = -2X
$$
\noindent 对于C则有
$$
\frac{\partial \rho_C}{\partial t} + \nabla \cdot (\rho_C \vec v_C) = 3X
$$
\noindent 可以看到与总的质量守恒方程式(1)自洽的。另外值得说明的是,
这里引入了一个新的场函数$X$,用以描述空间中化学反应进行的情况。从化学反应的角度考虑,$X$可以是
由$\rho_A,\rho_B,\rho_C,\vec v_A,\vec v_B,\vec v_C$以及温度等场函数共同决定的函数。

\section{第三题}

记模型中的一维坐标为$x$,引入沿$x$坐标方向的应力$p(x)$,由于一维模型,
可以认为质量力也只有沿着$x$坐标方向的分量,记为$F(x)$,截面面积大小记为$A(x)$,
则在$x$和$x+\Delta x$取各取一个截面,和管道围成一个体积$V$,利用方程
$$
\int_V \frac{d\vec v}{dt} \rho dV = \int_V \vec F \rho dV + \int_{\partial V} \vec p \cdot d\vec \sigma
$$
$$
\Rightarrow
\int_V \frac{d\vec v}{dt} \rho A dx = \int_V \vec F \rho A dx + \int_{\partial V} \vec p \cdot d\vec \sigma
$$
\noindent 并且用到积分的中值定理,可以得到
$$
\dot v(\xi) \rho(\xi) A(\xi) \Delta x = F(\eta ) \rho(\eta) A(\eta) \Delta x + (-p(x+\Delta x)A(x+\Delta x)+p(x)A(x))
$$
\noindent 其中$\xi,\eta$是$x$和$x+\Delta x$之间的某两个点,上式两边除以$\Delta x$并令$\Delta x\rightarrow 0$即得
\begin{equation}
\rho A \dot v=\rho A F - \frac{d(pA)}{dx} \tag{1}
\end{equation}
\end{document}
