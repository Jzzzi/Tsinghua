\documentclass[UTF8]{ctexart}
\usepackage{amsmath}
\usepackage{amssymb}
\usepackage{cases}
\usepackage{cite}
\usepackage{graphicx}
\usepackage[margin=1in]{geometry}
\geometry{a4paper}
\usepackage{lmodern}%处理字体问题

\title{连续介质力学作业六}
\author{刘锦坤\\2022013352}
\date{\today}

\begin{document}
\maketitle

首先作者在关于标号的写法上存在问题,文章中有提到两组新老坐标$x^i$和$x^{i'}$,
这种标号的使用是不准确且易混淆的,最好是以以$x^i$和$x'^i$表示之。
事实上这种标号的使用错误通篇皆是,例如$\boldsymbol g'^i$写为了$\boldsymbol g^{i'}$等等,
这不仅使得读者阅读困难,也让人怀疑原作者本人到底是否理解这样标号的含义。

如果说标号的写法尚可归入笔误的范畴,从原文中的方程(8)开始使用的求和约定就可谓是匪夷所思了。
\begin{equation}
    \begin{aligned}
    &dx^i=\frac{\partial x^i}{\partial x^{i'}}dx^{i'} \triangleq \beta^i_{i'}dx^{i'}\\
    &dx^{i'}=\frac{\partial x^{i'}}{\partial x^i}dx^i \triangleq \beta^{i'}_{i}dx^i
    \end{aligned}
    \tag*{(8)}
\end{equation}
这个式子是不容易理解的,一个表达式中竟然出现了3个$i$指标,这是一种什么样的含义呢?
亦或是作者认为原本区分新旧坐标架的$'$标号在此刻决心与$i$结合而成为一个全新的$i'$标号?
无从揣测作者的真实想法,但大概可以修改为一个常人尚能理解的形式。
\begin{equation}
    \begin{aligned}
    &dx^i=\frac{\partial x^i}{\partial x'^{j}}dx'^{j}\\
    &dx'^i=\frac{\partial x'^i}{\partial x^j}dx^j
    \end{aligned}
    \tag*{(8')}
\end{equation}
至于原作者关于$\beta$这个量的定义则实在是让人难以明白,这里就不冒昧的再次引入这个量了,
不妨在之后的运算中就写作
$\frac{\partial x^i}{\partial x'^{j}}$和$\frac{\partial x'^i}{\partial x^j}$。

然后是作者提出了一个新的概念,即“纯化公设”。引用作者的原文即是
\begin{quote}
    “一个张量,要么只能被表达在老基矢量下,要么只能被表达在新基矢量下。”
\end{quote}
首先,以二阶张量为例,不知原作者有何高见可以用一组基矢量表出一个二阶张量。
从常理来看,人们似乎习惯于把一个二阶张量利用基矢量的并积来表出,而不是单独的基矢量表出。
而原作者的意思似乎是,除了他之外的所有人都认为这些并积一定是同一组坐标系的基矢量之间的并积。
除了他之外的所有人也都仅仅只会将张量展开为这种并积的线性组合。
作者自鸣得意地将他这一伟大的发现称为“纯化公设”,并且在之后的文章中大肆宣扬。
事实上,就我的学习经历来看,我似乎从未学习过一种这样的“纯化公设”,
而且似乎只需做一点微不足道的数学变化就可以打破这种“纯化公设”,例如
\begin{equation}
    \boldsymbol {T}
    = T^{ij}\boldsymbol {g}_i \boldsymbol {g}_{j}
    =T^{ij}\boldsymbol {g}_i \frac{\partial x'^k}{\partial x^j}\boldsymbol {g'}_{k}
    =T^{ij} \frac{\partial x'^k}{\partial x^j} \boldsymbol {g}_i \boldsymbol {g'}_{k}
    \notag{}
\end{equation}
三个等号就可打破的“公”设,实在让人怀疑这“公”设究竟经过多少学者的“公”认。

暂且不论作者缘何产生这种自己孑立于历代学者之外的想法。
先来看看作者在打破“纯化公设”后作出的伟大结论,即关于杂交分量的部分。
其实完全无需作者通篇复杂啰嗦的计算,一行等式足以给出所谓度量张量的杂交分量。
\begin{equation}
    \boldsymbol{G}
    =\boldsymbol{g}_i \boldsymbol{g}^i
    =\frac{\partial x'^k}{\partial x^i} \boldsymbol{g'}_k \boldsymbol{g}^i
    \notag{}
\end{equation}
作者大费笔墨得出的所谓度量张量的杂交分量,其实作简单的一行数学变换就可以得到,实在是让人不知所云。

至于原作者在8.9.两节中提到的所谓“指标变换的统一”,更是毫无意义,不值一提。
其本质不过就是再简单不过的偏导的链式法则,不知作者特意再次强调这种指标变换有何意义。

不过作者在10.中的推导就不是那么显然的了,这节中定义了一种作用于所谓杂交分量的$\nabla$算子。
从数学上看,这种$\nabla$算子的定义当然是可行的。但是在这一节中,
作者的推导由于其紊乱的符号系统,基本不具有可读性。尤其是在打破了所谓的“纯化公设”之后,
对于一个用两个不同坐标系($x^i,x'^i$)基矢量的并积展开的张量,
又只对其中一个坐标(例如$x^i$)求偏导,这就导致其运算结果本身非常繁琐复杂。
而且作者也没有指明这样一个不伦不类的杂交Christoffel符号有何意义。

至于11.中关于两点张量的论述,方程组(51)(52)(53)本质上就是普通的坐标变换,
而之前作者定义的$\beta$也不过是坐标变换的Jacobi矩阵中的元素,二者自然是一致的。
至于所谓Riemann空间中情况,也并不明白作者希望由此表达一个什么含义,

总之,这篇文章从头到尾采用着紊乱的符号规则,又臆断性地提出了所谓的“纯化公设”,
随后又打破自己臆断出的这种“纯化公设”。整篇文章不明所以,
至于作者在之后的章节自鸣得意间发表的感言总结更是惨不忍睹。就此作为一篇论文发布并不合适,也不合格。
\end{document}