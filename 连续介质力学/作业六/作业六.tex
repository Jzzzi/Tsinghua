\documentclass[UTF8]{ctexart}
\usepackage{amsmath}
\usepackage{amssymb}
\usepackage{cases}
\usepackage{cite}
\usepackage{graphicx}
\usepackage[margin=1in]{geometry}
\geometry{a4paper}
\usepackage{lmodern}%处理字体问题

\title{连续介质力学作业六}
\author{刘锦坤\\2022013352}
\date{\today}

\begin{document}
\maketitle

首先作者在关于标号的写法上存在问题,文章中有提到两组新老坐标$x^i$和$x^{i'}$,
这种标号的使用是不准确且易混淆的,最好是以以$x^i$和$x'^i$表示之。
事实上这种标号的使用错误通篇皆是,例如$\boldsymbol g'^i$写为了$\boldsymbol g^{i'}$等等,
这不仅使得读者阅读困难,也让人怀疑原作者本人到底是否理解这样标号的含义。

如果说标号的写法尚可归入笔误的范畴,从原文中的方程(8)开始使用的求和约定就可谓是匪夷所思了。
\begin{equation}
    \begin{aligned}
    &dx^i=\frac{\partial x^i}{\partial x^{i'}}dx^{i'} \triangleq \beta^i_{i'}dx^{i'}\\
    &dx^{i'}=\frac{\partial x^{i'}}{\partial x^i}dx^i \triangleq \beta^{i'}_{i}dx^i
    \end{aligned}
    \tag*{(8)}
\end{equation}
这个式子是不容易理解的,一个表达式中竟然出现了3个$i$指标,这是一种什么样的含义呢?
亦或是作者认为原本区分新旧坐标架的$'$标号在此刻决心与$i$结合而成为一个全新的$i'$标号?
无从揣测作者的真实想法,但大概可以修改为一个常人尚能理解的形式。
\begin{equation}
    \begin{aligned}
    &dx^i=\frac{\partial x^i}{\partial x'^{j}}dx'^{j}\\
    &dx^{i'}=\frac{\partial x'^i}{\partial x^j}dx^j
    \end{aligned}
    \tag*{(8)}
\end{equation}
\end{document}