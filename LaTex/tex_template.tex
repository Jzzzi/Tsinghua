% MIT License

% Copyright (c) 2022 Chiyuru

% Permission is hereby granted, free of charge, to any person obtaining a copy of this software and associated documentation files (the "Software"), 
% to deal in the Software without restriction, including without limitation the rights
% to use, copy, modify, merge, publish, distribute, sublicense, and/or sell
% copies of the Software, and to permit persons to whom the Software is
% furnished to do so, subject to the following conditions:

% The above copyright notice and this permission notice shall be included in all copies or substantial portions of the Software.

% THE SOFTWARE IS PROVIDED "AS IS", WITHOUT WARRANTY OF ANY KIND, EXPRESS OR IMPLIED, INCLUDING BUT NOT LIMITED TO THE WARRANTIES OF MERCHANTABILITY,
% FITNESS FOR A PARTICULAR PURPOSE AND NONINFRINGEMENT. IN NO EVENT SHALL THE AUTHORS OR COPYRIGHT HOLDERS BE LIABLE FOR ANY CLAIM, DAMAGES OR OTHER
% LIABILITY, WHETHER IN AN ACTION OF CONTRACT, TORT OR OTHERWISE, ARISING FROM,
% OUT OF OR IN CONNECTION WITH THE SOFTWARE OR THE USE OR OTHER DEALINGS IN THE SOFTWARE.

\documentclass[UTF8]{ctexart}

\usepackage{amsmath}
\usepackage{cases}
\usepackage{cite}
\usepackage{graphicx}
\usepackage[margin=1in]{geometry}
\geometry{a4paper}
\usepackage{fancyhdr}
\pagestyle{fancy}
\fancyhf{}


\title{基础物理实验报告}
\author{\LaTeX\ by\ 驰雨Chiyuru}
\date{\today}
\pagenumbering{arabic}

\begin{document}

%\fancyhead[L]{驰雨Chiyuru}
\fancyhead[C]{旋转摆LQR平衡控制}
\fancyfoot[C]{\thepage}

\maketitle
\tableofcontents
\newpage

\section{摘要}
简要概述主要实验内容和结果。要求:使用标准精确的词汇和语言,清晰紧凑地概述客观事实;摘要的整体结构严谨、思路清楚,基本素材组织合理。英文摘要与中文内容一致。论文的中、英文摘要是国内外数据库收录的主要内容,所以摘要的内容直接影响到该论文能否被收录及收录后被引用的情况,作者应给予高度重视。


\section{实验仪器}
要对使用的实验仪器做出一些解释,每个部件用来做什么,怎么操作,操作原理为何。文中引用的结论性文字要标注参考文献,须加方括号,一般置于右上角。如\cite{王合英2018自主探究实验对学生综合素质和创新能力的培养}

\subsection{实验仪器1}
实验仪器1的结构如图$1$所示。部件$A$与部件$B$连接,构成xx系统,固定在部件$C$上的点$1$位置。使用原理是xxxxxx。


\subsection{实验仪器2}
实验仪器2使用方法可参考说明书。


\section{实验原理}

\subsection{xxx方程}
在xx,xxx,xxxx条件下,考察条件为xx的xx的情况,利用xxxx定律在无位移的水平方向和有位移的竖直方向分别列出以下方程:


\begin{numcases}{}
    T_2cos\alpha_2 - T_1cos\alpha_1 = 0 \\
    T_2sin\alpha_2 - T_1sin\alpha_1 = \rho dx\frac{\partial^2y }{\partial x^2} 
\end{numcases}

\subsection{xxx情况下的边界条件和xx现象}
xxxx时发生xxxx现象。由xxx方程可知,xxx波形为$y^+=f(vt+x)$,xxx波形为$y^-=f(vt-x)$。

\subsection{xx在xxx条件下的xxx现象}
Complete by yourself!


\section{实验过程与数据分析}
\subsection{A.在xx条件下测量xxx}
\subsubsection{$a1. $计算出xx的电阻和电感}
在xx上将xx的两端串联xx和xx相连,将xx的两端串联进xx,分别将xx接在$L_1$,$L_2$,xx的两端测量xx并记录。
\subsubsection{$a2. $Complete by yourself!}
Complete by yourself!
\subsubsection{$a3. $Complete by yourself!}
实验得到的数据如下:

\begin{center}
\begin{tabular}{|c|c|c|c|c|c|}
 \hline
线圈名称 & R'(Ω) & Va(V) & V(V) & Vr'(V) & Vo(V)\\
 \hline
线圈1(空气芯) & 123 & 456 & 789 & 012 & 345\\
 \hline
线圈2(空气芯) & 123 & 456 & 789 & 012 & 345\\
 \hline
线圈3(铝芯) & 123 & 456 & 789 & 012 & 345\\
 \hline
线圈4(铝芯) & 123 & 456 & 789 & 012 & 345\\
 \hline
\end{tabular}
\end{center}

\subsection{展示一下行间公式}
\subsubsection{行间公式}
% 行间公式用 $$ $$ 或者 \[ \] 来框住都可以,但在 LaTeX 中前者会改变行文的默认行间距,因此不推荐采用。
\paragraph{}这是一个不确定度计算。
\[
U_k=tinv(x,y)×s_k=xxx
\]
\subsubsection{相对于行内公式}
这是一个不确定度计算:$U_k=tinv(x,y)×s_k=xxx$


\section{分析与讨论}

\subsection{误差分析}

\subsubsection{实验中的系统误差}
来自xxx的精度影响。

受空间内xx与xx的干扰。

\subsubsection{实验中的偶然误差}
接线时可能有xxx情况,导致xxx。xx上的xx在某情况下有xx的问题存在,经反复调整后得以正常测量。

\subsection{实验后的思考}
可说明自己做本实验的总结、收获和体会,对实验中发现的问题提出自己的建议。

\newpage
%图一般很大,建议换页。
\section{原始数据}
\begin{center}
    Change the picture by yourself!
    
    
%    \includegraphics{picture/example.png}
\end{center}



\bibliographystyle{plain}
\bibliography{./template}  %bib文件名

\end{document}